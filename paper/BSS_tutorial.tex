%%%%%%%%%%%%%%%%%%%%%%%%%%%%%%%%%%%%%%%%%%%%%%%%%%%%%%%%%%%%
%%% LIVECOMS ARTICLE TEMPLATE FOR BEST PRACTICES GUIDE
%%% ADAPTED FROM ELIFE ARTICLE TEMPLATE (8/10/2017)
%%%%%%%%%%%%%%%%%%%%%%%%%%%%%%%%%%%%%%%%%%%%%%%%%%%%%%%%%%%%
%%% PREAMBLE
\documentclass[9pt,tutorial]{livecoms}
% Use the 'onehalfspacing' option for 1.5 line spacing
% Use the 'doublespacing' option for 2.0 line spacing
% Use the 'lineno' option for adding line numbers.
% Use the "ASAPversion' option following article acceptance to add the DOI and relevant dates to the document footer.
% Use the 'pubversion' option for adding the citation and publication information to the document footer, when the LiveCoMS issue is finalized.
% The 'bestpractices' option for indicates that this is a best practices guide.
% Omit the bestpractices option to remove the marking as a LiveCoMS paper.
% Please note that these options may affect formatting.

\usepackage{lipsum} % Required to insert dummy text
\usepackage[version=4]{mhchem}
\usepackage{siunitx}
\DeclareSIUnit\Molar{M}
\usepackage[italic]{mathastext}
\graphicspath{{figures/}}
\usepackage{hyperref}
%%%%%%%%%%%%%%%%%%%%%%%%%%%%%%%%%%%%%%%%%%%%%%%%%%%%%%%%%%%%
%%% IMPORTANT USER CONFIGURATION
%%%%%%%%%%%%%%%%%%%%%%%%%%%%%%%%%%%%%%%%%%%%%%%%%%%%%%%%%%%%
\usepackage[colorinlistoftodos]{todonotes}
\newcommand{\versionnumber}{1.0}  % you should update the minor version number in preprints and major version number of submissions.
\newcommand{\githubrepository}{\url{https://github.com/myaccount/homegithubrepository}}  %this should be the main github repository for this article

%%%%%%%%%%%%%%%%%%%%%%%%%%%%%%%%%%%%%%%%%%%%%%%%%%%%%%%%%%%%
%%% ARTICLE SETUP
%%%%%%%%%%%%%%%%%%%%%%%%%%%%%%%%%%%%%%%%%%%%%%%%%%%%%%%%%%%%
\title{BioSimSpace Tutorial [Article v\versionnumber]}

\author[1*]{Antonia S J S Mey}
\author[2]{Lester Hedges}
\author[1]{Sofia Bariami}
\author[1]{Julien Michel}
\author[2]{Christopher Woods}
\affil[1]{Edinburgh}
\affil[2]{Institution 2}

\corr{antonia.mey@ed.ac.uk}{AM}  % Correspondence emails.  FMS and FS are the appropriate authors initials.
\corr{email2@example.com}{FS}

\orcid{Antonia S J S Mey}{0000-0001-7512-5252}
\orcid{Author 2 name}{EEEE-FFFF-GGGG-HHHH}

\contrib[\authfn{1}]{These authors contributed equally to this work}
\contrib[\authfn{2}]{These authors also contributed equally to this work}

\presentadd[\authfn{3}]{Department, Institute, Country}
\presentadd[\authfn{4}]{Department, Institute, Country}

\blurb{This LiveCoMS document is maintained online on GitHub at \githubrepository; to provide feedback, suggestions, or help improve it, please visit the GitHub repository and participate via the issue tracker.}

%%%%%%%%%%%%%%%%%%%%%%%%%%%%%%%%%%%%%%%%%%%%%%%%%%%%%%%%%%%%
%%% PUBLICATION INFORMATION
%%% Fill out these parameters when available
%%% These are used when the "pubversion" option is invoked
%%%%%%%%%%%%%%%%%%%%%%%%%%%%%%%%%%%%%%%%%%%%%%%%%%%%%%%%%%%%
\pubDOI{10.XXXX/YYYYYYY}
\pubvolume{<volume>}
\pubissue{<issue>}
\pubyear{<year>}
\articlenum{<number>}
\datereceived{Day Month Year}
\dateaccepted{Day Month Year}

%%%%%%%%%%%%%%%%%%%%%%%%%%%%%%%%%%%%%%%%%%%%%%%%%%%%%%%%%%%%
%%% ARTICLE START
%%%%%%%%%%%%%%%%%%%%%%%%%%%%%%%%%%%%%%%%%%%%%%%%%%%%%%%%%%%%

\begin{document}

\begin{frontmatter}
\maketitle

\begin{abstract}
BioSimSpace is an interoperable molecular simulation framework providing a uniform Python API for common molecular simulation packages such as Gromacs, NAMD, SOMD, and AMBER. This tutorial introduces different aspects of the API from basic simulation setup to more advanced types of simulations such as alchemical free energy simulations, as well as metadynamics simulations. 
\end{abstract}

\end{frontmatter}



%%%%%%%%%%%%%%%%%%%%%%%%%%%%%%
%%%%% Introduction
%%%%%%%%%%%%%%%%%%%%%%%%%%%%%%
\section{Introduction}
\label{sec:intro}
Molecular dynamics (MD) simulations are a widely used tool to get atomistic insight into the behavior of biological and soft matter systems, making important contributions to structural biology, pharmaceutical research, as well as material sciences~\cite{huggins2019biomolecular, rountree2002atomistic}. A vast array of different software tools are available to run and analyse MD simulations. This means as a new practitioner it is oftentimes quite difficult to get started with what software package to use for what scientific question. 
BioSimSpace(BSS) was designed in aid to alleviate this problem by trying to provide an interoperable versatile software frame work that can be easily adapted for the specific purpose required~\cite{hedges2019biosimspace}. It provides a common python API that will allow to run and analyse different types of MD simulations in a simulation platform agnostic way. This means that the same python script or command line tool can be used to run a simulation using Gromacs~\cite{vanderspoel2005gromacs}, AMBER~\cite{}, OpenMM~\cite{eastman2017openmm}, or NAMD~\cite{phillips2005scalable}. As part of BSS it is also possible to readily convert between different file formats supported across these common simulation platforms. Furthermore, it also aims to implement common best practices for setting simulation parameters making it less confusing for new practitioners who may not understand the zoo of possible parameters that can be set for a given simulation package. Another aspect is the level of abstraction introduced to run complex simulation such as alchemical free energy calculations~\cite{} of binding or metadynamics simualtions~\cite{}. This document aims to give an introduction on how to use BSS in conjunction with different software packages to perform different types of molecular dynamics simulation of varying difficulty.

%%%%%%%%%%%%%%%%%%%%%%%%%%%%%%
%%%%% Scope
%%%%%%%%%%%%%%%%%%%%%%%%%%%%%%
\section{Scope}
\label{sec:scope}
The document is divided into three tutorial sections. The basic BSS tutorial section~\ref{sec:basic_tutorial} introduces the software, how to run it using different interfaces such as a command-line approach or within a Jupyter notebook. In section~\ref{sec:basic_tutorial} the user is slowly introduced to concepts of different complexity, starting with a simple question such as, how can I convert my Gromacs files, such that I can run them with AMBER convert in sec.~\ref{subsec:conversion}. Then through a series of steps the user is shown how to setup (sec.~\ref{subsec:setup} and run~\ref{subsec:running_MD} a basic molecular dynamics simulation. These steps are:
\begin{enumerate}
\item Parametrise the system 
\item Solvate the system
\item minimise the system
\item Equilibration
\item Running a molecular dynamics simulation of the solvated box
\end{enumerate}
After having completed the sec. ~\ref{subsec:setup} the user should be able to setup an unknown small molecule or protein in a solvent box for any of the supported MD simulation engines, either using a command-line interface or a Jupyter notebook. They should be able to adjust simple simulation parameters such as the number of minimisation steps, or how long the simulation should be run for. The user should also be able understand how to do some basic troubleshooting with provided error messages. 
Sec.~\ref{subsec:running_MD} illustrates how to actually run an MD simulation, adjust simulation lengths and reporting intervals. It also covers how simulations can be monitored 'on the fly' in a jupyter notebook environment with simple graph analyses and snapshots. 



%%%%%%%%%%%%%%%%%%%%%%%%%%%%%%
%%%%% Prerequisites
%%%%%%%%%%%%%%%%%%%%%%%%%%%%%%
\section{Prerequisites}
\label{sec:prerequisites}
This tutorial was written with BioSimSpace version 2019.3.0 in mind and the user is expected to have installed BioSimSpace version 2019.3.0. 

\subsection{Installations, Dependencies and Issues}
\label{subsec:software}
For help with the installation, see the online documentation:~\url{https://biosimspace.org/install.html}, and information on compatibility of different MD packages can be found at~\url{https://biosimspace.org/compatibility.html}. 
In order to obtain CUDA support for OpenMM, an OpenMM supported version of CUDA drivers and CUDA toolkit need to be installed. For convenience, try running \texttt{optimise\_openmm}, a script used to sync OpenMM  and CUDA version for compatibility. For any issues with the tutorial or BioSimSpace, please use the issue tracker on Github (\url{https://github.com/michellab/BioSimSpace/issues}). 

\subsection{Background knowledge}
\label{subsec:background}
The tutorials can be run in different ways, either as command-line tools from a Unix terminal, or as part of a Jupyter notebook, or simply on the provided cloud service, meaning that no installation is required. A basic familiarity with Unix command-line tools or Jupyter notebooks is assumed to have the best benefit of the tutorials. Since BSS is primarily focussed on facilitating the setup and running of MD simulations the reader is encouraged to engage in some background reading on best practices for running molecular dynamics simulations. The following are a none exhaustive list of papers and books that serve as a good starting point for understanding molecular simulations~\cite{braun2019best, frenkel2001understanding, alavi2011statisticala, leimkuhler2015molecular, rapaport2004art}. However, it is not required to be familiar with any of the MD simulations packages supported, as BSS provides an abstraction layer that does not require in depth knowledge of any of the simulation package, unless you want to learn about non-standard simulations supported by individual packages, which is out of scope of this tutorial. 



%%%%%%%%%%%%%%%%%%%%%%%%%%%%%%
%%%%% Prerequisites
%%%%%%%%%%%%%%%%%%%%%%%%%%%%%%
\section{Content and links}
\label{sec:content}
All files needed for the tutorial can be found at \url{my/custom/url}

%%%%%%%%%%%%%%%%%%%%%%%%%%%%%%
%%%%% Basic tutorials
%%%%%%%%%%%%%%%%%%%%%%%%%%%%%%
\section{Basic BioSimSpace tutorials}
\label{sec:basic_tutorial}
BioSimSpace has a very versatile Python API that allows not just to write typical molecular dynamics simulation and analysis workflow, but can also be used for simple tasks such as converting between different MD file types. In the basic tutorial section file conversions, molecular simulation setup, including parametrisation, minimisation and equilibration as well as running MD simulations will be covered. \\

\textbf{@@@@@@@@@@@ Sofia inserted:} \\
At the tutorials to follow, the Jupyter notebook layout will be used, and the Node object will be the core of our molecular workflow component. Nodes define the input that is needed, and the output that is produced.

\subsection{Conversions}
\label{subsec:conversion}
\todo[inline, color={red!20}]{@Toni please write section}

\subsection{Simulation setup}
\label{subsec:setup}

The first and certainly most important step to run an MD simulation is to set the system correctly. This set up is typically consisted of three steps: i) The system parameterisation and solvation, ii) the energy minimisation and iii) the system equilibration. More specifically, the system of interest should be parameterised with the appropriate forcefield and then solvated. To remove any potential intermolecular atom clashes and minimise the forces between the atoms, an energy minimisation should be conducted. Finally with NVT and NPT equilibration the system is relaxed, the solvent molecules are rearranged around the solute, and both the temperature and the pressure are set to the desirable values. At the following sub-sections we will show how to run these processes with BSS. At the end, we will have a robust and interoperable workflow to generate a molecular system ready for simulation with AMBER.

\subsubsection{Parametrisation and Solvation}
\todo[inline, color={red!20}]{@Sofia please write section}
In this tutorial the parameterisation and solvation of Hsp90 alpha ATPase domain (2JJC.pdb) will be demonstrated. We are going to use the ff14SB forcefield and the TIP3P water model. \\

First we need to import BioSimSpace, create a Node object and set the author and license. 

\begin{verbatim}
import BioSimSpace as BSS
node = BSS.Gateway.Node("A node to parameterise and 
solvate a molecule ready for molecular simulation
with AMBER.")

node.addAuthor(name="Lester Hedges", 
email="lester.hedges@bristol.ac.uk", 
affiliation="University of Bristol")

node.setLicense("GPLv3")
\end{verbatim}
Nodes require inputs. To specify inputs we use the Gateway module, which is used as a bridge between BioSimSpace and the outside world. This will allow us to document the inputs, define their type, and specify any constraints on their allowed values.
\begin{verbatim}
node.addInput("file", BSS.Gateway.File(help="A 
molecular input file, e.g. a PDB file."))

node.addInput("forcefield", BSS.Gateway.String(help=
"The name of the force field to use for param/tion.",
allowed=BSS.Parameters.forceFields(),
default="ff14SB"))
 
node.addInput("water", BSS.Gateway.String(help=
"The name of the water model to use for solvation.", 
allowed=BSS.Solvent.waterModels(), default="tip3p"))

node.addInput("box_size", BSS.Gateway.Length(help=
"The base length of the cubic simulation box.",
unit="nanometer"))

node.addInput("ion_conc", BSS.Gateway.Float(help="The
ionic concentration in mol/litre.",minimum=0, 
maximum=1, default=0))
\end{verbatim}

We also need to define the output of the node. In this case we will return a set of AMBER format files for the parameterised and solvated system.

\begin{verbatim}
node.addOutput("system", BSS.Gateway.FileSet(help=
"The parameterised and solvated molecular system 
in AMBER format."))
\end{verbatim}

The input files can be set with the help of the Graphical User Interface (GUI) that Jupyter notebooks provide. The \texttt{node.showControls()}
method, will display this GUI from which inputs can be set actively by the user. All input requirements that don't have a default value must be set before the node can proceed. An error will be raised if you try to query the node for one of the user values. Files from your local system can be uploaded through the GUI. Jupyter has a limit of 5MB for file transfers, but support for compressed formats, such as \texttt{.zip} or \texttt{.tar.gz} is provided. The input files can then be accessed using the \texttt{node.getInput()} method. The first time this is called the node will automatically validate all of the input and report the user if any errors were found. \\

BioSimSpace has support for a wide range of formats and can convert between certain formats too. It can also automatically determine the format of the file that is uploaded, so we do not need to specify it. To create a molecular system we just need to call the \texttt{readMolecules()} function: 

\begin{verbatim}
system = BSS.IO.readMolecules(node.getInput("file"))
\end{verbatim}

Now, we need to grab the molecule from the system and parameterise it. We do so by calling the parameterise function from the BSS.Parameters package, passing the molecule and force field name as arguments. Since parameterisation can be slow, the function returns a handle to a process that runs the parameterisation in the background. To get the parameterised molecule from the process we need to call the \texttt{getMolecule} method. This is a blocking operation which waits for the process to finish before grabbing the molecule and returning it.

\begin{verbatim}
molecule = system[0]
molecule = BSS.Parameters.parameterise(molecule,
node.getInput("forcefield")).getMolecule()
\end{verbatim}

Next we will solvate the molecule in a box of water using the \texttt{solvate} function from the BSS.Solvent package. This will centre the molecule in a cubic box of a specified size and surround it by water molecules. The user can specify the size of the box and the ionic strength. By default, the resulting system will also be neutralised.

\begin{verbatim}
system = BSS.Solvent.solvate(node.getInput("water"),
molecule=molecule,box=3 *[node.getInput("box_size")],
ion_conc=node.getInput("ion_conc"))
\end{verbatim}

To save the solvated system, we will use the \texttt{saveMolecules()} function. This function allows the user to define the file format of the output. Here we will return a set of AMBER format files representing the parameterised and solvated molecular system. 

\begin{verbatim}
node.setOutput("system", BSS.IO.saveMolecules
("system", system, ["prm7", "rst7"]))
\end{verbatim}

Finally, we validate that the node completed successfully. This will check that all output requirements are satisfied and that no errors were raised by the user. Any file outputs will be available for the user to download as a compressed archive.

\begin{verbatim}
node.validate()
\end{verbatim}

Once we are satisfied with our node we can choose to download it as a regular Python script that can be run from the command-line, by clicking on: \texttt{File/Download As/Python}

\subsubsection{Minimisation}
\todo[inline, color={red!20}]{@Sofia please write section}

Since we have parameterised and solvated our molecular system, energy minimisation is next on the list. Given that we have already set our files, as described at the previous subsection, we should now set an integer that indicates the number of minimisation steps to perform. Note that he input requirement steps has a default value, so it is optional.

\begin{verbatim}
node.addInput("steps", BSS.Gateway.Integer(help=
"The number of minimisation steps.", minimum=0,
maximum=1000000, default=10000))
\end{verbatim}

We now need to define the output of the node. In this case we will return a set of files representing the minimised molecular system.

\begin{verbatim}
node.addOutput("minimised", BSS.Gateway.FileSet(help=
"The minimised molecular system."))
\end{verbatim}

In order to run a minimisation, a protocol needs to be defined. When defining a protocol we are configuring the type of simulation that we wish to run, as well as any options for the particular simulation. This can be done using the \texttt{BioSimSpace.Protocol} module.  Then, to run the protocol, we need to find an appropriate molecular dynamics package on our current environment. What package is found will depend upon both the system and protocol, as well as the hardware that is available to the user. We will use \texttt{BioSimSpace.MD} to find the appropriate MD package: 

\begin{verbatim}
protocol = BSS.Protocol.Minimisation
(steps=node.getInput("steps"))
process = BSS.MD.run(system, protocol)
\end{verbatim}

We can get the minimised molecular configuration once the process has finished running. This will be saved to a file using the same format as the original system, and the name that we provide. Note that we pass \texttt{block=True} to \texttt{getSystem}. Since we're working interactively the process is running in the background and \texttt{getSystem} will grab and return the latest molecular configuration. By passing \texttt{block=True} we wait for the process to finish before getting the final configuration and writing it to file. 

\begin{verbatim}
node.setOutput("minimised", BSS.IO.saveMolecules
("minimised", process.getSystem(block=True),
system.fileFormat()))
\end{verbatim}

Finally, we validate again that the node completed successfully, as we have done after the solvation of the system.

\begin{verbatim}
node.validate()
\end{verbatim}

\subsubsection{Equilibration}
\todo[inline, color={red!20}]{@Sofia please write section}

Before running the final MD simulation, we need to equilibrate our molecular system to a target temperature. Following a similar process as the one for minimisation, we need to define an equilibration protocol. To initialise the default BSS equilibration protocol, we call \texttt{protocol = BSS.Protcol.Equilibration()}: 
\begin{verbatim}
protocol = BSS.Protocol.Equilibration(
runtime=0.05*BSS.Units.Time.nanosecond, 
temperature_start=0*BSS.Units.Temperature.kelvin, 
temperature_end=300*BSS.Units.Temperature.kelvin, 
restrain_backbone=True)
\end{verbatim}

We now have everything that is needed to create a process object.
\begin{verbatim}
process = BSS.MD.run(system, protocol)
\end{verbatim}

The time, temperature, and energy can be monitored as the process runs: 
\begin{verbatim}
print(process.getTime(), process.getTemperature(),
process.getTotalEnergy())
\end{verbatim}
We can also check if the process is running: \texttt{process.isRunning()} and  how many minutes the process has been running for: \texttt{process.runTime()}.

After the process has finished running, we can save the equilibrated system and finally validate the node: 
\begin{verbatim}
node.setOutput("equilibrated", BSS.IO.saveMolecules
("equilibrated", process.getSystem(block=True),
system.fileFormat()))
node.validate()
\end{verbatim}

\subsection{Running and analysing MD simulations}
\label{subsec:running_MD}
\todo[inline, color={red!20}]{@Toni please write section}

%%%%%%%%%%%%%%%%%%%%%%%%%%%%%%
%%%%% Advanced tutorials
%%%%%%%%%%%%%%%%%%%%%%%%%%%%%%
\section{Advanced BioSimSpace tutorials}
\label{sec:advanced_tutorial}

\subsection{Alchemical Free Energy calculations}
\todo[inline, color={red!20}]{@Toni please write section}
\todo[inline, color={red!20}]{@Sofia, @Cameron please review section}

\subsection{Metadynamics}
\todo[inline, color={red!20}]{@Lester/Toni/Dom please write section}

%%%%%%%%%%%%%%%%%%%%%%%%%%%%%%
%%%%% API specifics
%%%%%%%%%%%%%%%%%%%%%%%%%%%%%%
\section{Working with the API}
\label{sec:API_tutorial}
\todo[inline, color={red!20}]{@Lester please write section}




\section*{Author Contributions}
ASJSM: Wrote the manuscript. 

% We suggest you preserve this comment:
For a more detailed description of author contributions,
see the GitHub issue tracking and changelog at \githubrepository.

\section*{Other Contributions}
%%%%%%%%%%%%%%%
% You should include all people who have filed issues that were
% accepted into the paper, or that upon discussion altered what was in the paper.
% Multiple significant contributions might mean that the contributor
% should be moved to authorship at the discretion of the a
%
% See the policies ``Policies on Authorship'' section of https://livecoms.github.io for
% more information on deciding on authorship and author order.
%%%%%%%%%%%%%%%

(Explain the contributions of any non-author contributors here)
% We suggest you preserve this comment:
For a more detailed description of contributions from the community and others, see the GitHub issue tracking and changelog at \githubrepository.

\section*{Potentially Conflicting Interests}
%%%%%%%
%Declare any potentially competing interests, financial or otherwise
%%%%%%%

Declare any potentially conflicting interests here, whether or not they pose an actual conflict in your view.

\section*{Funding Information}
%%%%%%%
% Authors should acknowledge funding sources here. Reference specific grants.
%%%%%%%
ASJSM, JM, LOH, and CW acknowledge funding through an EPSRC flagship software grant: EP/P022138/1

\section*{Author Information}
\makeorcid

\bibliography{bss_tutorial}

%%%%%%%%%%%%%%%%%%%%%%%%%%%%%%%%%%%%%%%%%%%%%%%%%%%%%%%%%%%%
%%% APPENDICES
%%%%%%%%%%%%%%%%%%%%%%%%%%%%%%%%%%%%%%%%%%%%%%%%%%%%%%%%%%%%

%\appendix


\end{document}
